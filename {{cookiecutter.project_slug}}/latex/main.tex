\documentclass[conference]{style/IEEEtran}
\IEEEoverridecommandlockouts
% The preceding line is only needed to identify funding in the first footnote. If that is unneeded, please comment it out.

\ifCLASSOPTIONcompsoc
\usepackage[caption=false, font=normalsize, labelfont=sf, textfont=sf]{subfig}
\else
\usepackage[caption=false, font=footnotesize]{subfig}
\fi

\usepackage{amsmath,amssymb,amsfonts}
\usepackage{algorithm}
\usepackage{algorithmic}
\renewcommand{\algorithmiccomment}[1]{\bgroup\hfill//~#1\egroup}
\usepackage{graphicx}
\usepackage{textcomp}
\usepackage{xcolor}

%\usepackage[hyphens]{url}
%\usepackage{hyperref}
%\Urlmuskip=0mu plus 1mu

\usepackage{tikz}
\usetikzlibrary{shapes.geometric, arrows, chains}
\def\layersep{1.5cm}
\DeclareMathAlphabet{\mathpzc}{OT1}{pzc}{m}{it}
\newcommand{\z}{\mathpzc{z}}

\usepackage[utf8]{inputenc}
\usepackage[english]{babel}

\usepackage[noadjust]{cite}
\renewcommand{\citepunct}{,\penalty\citepunctpenalty\,}
\renewcommand{\citedash}{--}% optionally
\bibliographystyle{IEEEtran}

\usepackage{pgfplots}
\pgfplotsset{compat=newest}
\usepgfplotslibrary{groupplots}
\usepgfplotslibrary{dateplot}

\newcommand{\abs}[1]{\textcolor{red}{[abs: #1]}}

\usepackage[nolist,nohyperlinks]{acronym}

%% Colors that we want to globally define and use
%% Consider using http://colorbrewer2.org/#type=sequential&scheme=BuGn&n=3 to get RGB values
\definecolor{color1}{RGB}{35,139,69}
\definecolor{color2}{RGB}{254,178,76} 
\definecolor{color3}{RGB}{254,217,118} 
\definecolor{color4}{RGB}{34,94,168} 
\definecolor{color5}{RGB}{252,197,192}
\definecolor{color6}{RGB}{228,26,28} 
\newcommand{\norm}[1]{\left\lVert#1\right\rVert}

\begin{document}
\bstctlcite{IEEEexample:BSTcontrol}

\title{ {{cookiecutter.paper_title}}
\thanks{The work of {{cookiecutter.first_author}} was supported by someone.}
}

\author{\IEEEauthorblockN{
	{{cookiecutter.first_author}} \IEEEauthorrefmark{1} and
	{{cookiecutter.second_author}}\IEEEauthorrefmark{2}
}%
	\IEEEauthorblockA{\IEEEauthorrefmark{1}{{cookiecutter.first_author_affiliation}}}%	
	\IEEEauthorblockA{\IEEEauthorrefmark{2}{{cookiecutter.second_author_affiliation}}}
}
	

\maketitle

\begin{abstract}
your abstract here
\end{abstract}

\begin{IEEEkeywords}'
ieee, keywords
\end{IEEEkeywords}

%% Go update the acronyms .tex file with everything
\begin{acronym}
	% Initialisms and Acronyms
	\acro{DPD}{digital predistortion}
	\acro{PA}{power amplifier}
	\acro{RF}{radio frequency}
	\acro{TX}{transmit}
	\acro{MIMO}{multiple-input multiple-output}
	\acro{NN}{neural network}
	\acro{ILA}{indirect learning architecture}
	\acro{LS}{least squares}
	\acro{ML}{machine learning}
	\acro{FIR}{finite impule responce}
	\acro{PSD}{power spectral density}
	\acro{EVM}{error vector magnitude}
	\acro{RNN}{recurrent neural network}
	\acro{MSE}{mean squared error}
	\acro{ACLR}{adjacent channel leakage ratio}
	\acro{C-RAN}{cloud radio access network}
	\acro{GPU}{graphics processing units}
	\acro{SDR}{software defined radio}
	\acro{FCC}{Federal Communications Commission}
	\acro{ADC}{analog-to-digital converter}
	\acro{DAC}{digital-to-analog converter}
	\acro{IBFD}{in-band full-duplex}
	\acro{ReLU}{rectified linear unit}
	\acro{PAPR}{peak-to-average power ratio}
	\acro{PE}{processing element}
	\acro{LUT}{lookup table}
	\acro{FF}{flip-flop}
	\acro{DSP}{digital signal processor}
\end{acronym} 

\section{Introduction}
Efficiently correcting nonlinearities  in \acp{PA} through \ac{DPD} is critical for enabling next-generation mobile broadband where there may be multiple \ac{RF} \ac{TX} chains arranged to form a massive \ac{MIMO} system~\cite{2014_MassiveMimo}, as well as new waveforms with bandwidths on the order of 100~MHz in the case of mmWave communications~\cite{2014_mmwave}. 


\section{Conclusions}
In this paper, we did some good science. Everyone should cite this paper.

\bibliography{refs}
\end{document}
